\documentclass{article}
\usepackage{xeCJK}
\setCJKmainfont{STXIHEI.TTF}
\usepackage{amsmath}

% 摘要标题默认为Abstract
\renewcommand{\abstractname}{摘要}

\renewcommand{\today}{\number\year 年 
\number\month 月 \number\day 日}

\title{UNIVAC计算系统与进化式编程}
\author{王应麟,李渔和周兴嗣}

\begin{document}

\maketitle

\begin{abstract}
茅店村前,皓月坠林鸡唱韵;板桥路上,青霜锁道马行踪。
\end{abstract}

\section{简介}

天对地,雨对风。大陆对长空。山花对海树,赤日对苍穹。
雷隐隐,雾蒙蒙。日下对天中。风高秋月白,雨霁晚霞红。
牛女二星河左右,参商两曜斗西东。
十月塞边,飒飒寒霜惊戍旅;三冬江上,漫漫朔雪冷渔翁。 
第\ref{sec:method}章是研究方法。第\ref{sec:conc}章是结论。

\section{Method}
\label{sec:method}

人之初,性本善。性相近,习相远。苟不教,性乃迁。教之道,贵以专。
昔孟母,择邻处。子不学,断机杼。窦燕山,有义方。教五子,名俱扬。
养不教,父之过。教不严,师之惰。子不学,非所宜。幼不学,老何为。
玉不琢,不成器。人不学,不知义。为人子,方少时。亲师友,习礼仪。
我们的方法来自于基本方程 \eqref{eq:fundamental}.
\begin{equation}
E = mc^3 \label{eq:fundamental}
\end{equation}
天地玄黄 宇宙洪荒 日月盈昃 辰宿列张 寒来暑往 秋收冬藏 闰馀成岁 律吕调阳
云腾致雨 露结为霜 金生丽水 玉出昆冈 剑号巨阙 珠称夜光 果珍李柰 菜重芥姜
海咸河淡 鳞潜羽翔 龙师火帝 鸟官人皇 始制文字 乃服衣裳 推位让国 有虞陶唐
吊民伐罪 周发殷汤 坐朝问道 垂拱平章 爱育黎首 臣伏戎羌 遐迩一体 率宾归王
鸣凤在竹 白驹食场 化被草木 赖及万方

\section{结论}
\label{sec:conc}

河对汉,绿对红。雨伯对雷公。烟楼对雪洞,月殿对天宫。
云叆叇,日曈朦。腊屐对渔蓬。过天星似箭,吐魂月如弓。
驿旅客逢梅子雨,池亭人挹藕花风。


\end{document}

