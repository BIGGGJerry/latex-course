\documentclass[12pt]{article}

\usepackage{url}

\usepackage{xeCJK}
\setCJKmainfont{STXIHEI.TTF}

\begin{document}
讲好科学类报告的十个秘密

作者:你自己

简介

这份练习的正文是Mark Schoeberl 和 Brian Toon 一篇好文的删减版:
\url{http://www.cgd.ucar.edu/cms/agu/scientific_talk.html}

我的秘密

这张“秘密”清单是我听过各色演讲后的个人总结。我不装专家——肯定有我没想到的诀窍。不过,你需要注意的90% 应该都在我这张清单上了。

1) 认真准备,理清思路。像讲故事一样。

2) 预演几遍。没有理由不好好准备。

3) 不要贪多。要想讲好,围绕一两个要点就够了。

4) 别列方程。据称,每增加一个方程,能听懂你讲座的人数就减少一半。也就是说,令 q 为你列的方程个数,n 为理解你讲座的人数,则

n = gamma 乘 (1/2) 的 q 次方

这里的比例 gamma 是一个常数。

5) 两三个结论足矣。像学术会议中间,讲座一个连一个,你的听众能记住两三点就不错了。

6) 跟听众说话,别跟屏幕说话。最常见的问题就是盯着投影仪念念叨叨。

7) 别发怪声。卡壳了也别用“嗯”“啊”做过渡。

8) 对图象要求高一点。我有这几点建议:

* 字号大一些。

* 追求简洁;不要放多余图片。

* 彩色更好。

9) 面对提问时友好一点。

10) 尽量用一点幽默。我也不明白,但专业报告上,再普通的笑话也变得好笑了。

\end{document}
