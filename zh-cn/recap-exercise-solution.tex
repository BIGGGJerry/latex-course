\documentclass[12pt]{article}

\usepackage{hyperref}
\usepackage{xeCJK}
\setCJKmainfont{STXIHEI.TTF}

\renewcommand{\labelenumi}{\arabic{enumi})}

\title{讲好科学类报告的十个秘密}
\author{你自己}

\begin{document}
\maketitle

\section{简介}

这份练习的正文是Mark Schoeberl 和 Brian Toon \href{http://www.cgd.ucar.edu/cms/agu/scientific_talk.html}{一篇好文}的删减版。

\section{我的秘密}

这张“秘密”清单是我听过各色演讲后的个人总结。我不装专家——肯定有我没想到的诀窍。不过,你需要注意的90\% 应该都在我这张清单上了。

\begin{enumerate}

\item 认真准备,理清思路。像讲故事一样。

\item 预演几遍。没有理由不好好准备。

\item 不要贪多。要想讲好,围绕一两个要点就够了。

\item 别列方程。据称,每增加一个方程,能听懂你讲座的人数就减少一半。也就是说,令 $q$ 为你列的方程个数,$n$ 为理解你讲座的人数,则
\begin{equation}
n = \gamma \left( \frac{1}{2} \right)^q
\end{equation}
这里的比例 $\gamma$ 是一个常数。

\item 两三个结论足矣。像学术会议中间,讲座一个连一个,你的听众能记住两三点就不错了。

\item 跟听众说话,别跟屏幕说话。最常见的问题就是盯着投影仪念念叨叨。

\item 别发怪声。卡壳了也别用“嗯”“啊”做过渡。

\item 对图象要求高一点。我有这几点建议:

\begin{itemize}
\item[*] 字号大一些。

\item[*] 追求简洁;不要放多余图片。

\item[*] 彩色更好。

\end{itemize}

\item 面对提问时友好一点。

\item 尽量用一点幽默。我也不明白啊,但专业报告上,再普通的笑话也变得好笑了。

\end{enumerate}

\end{document}
